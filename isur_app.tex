\documentclass[12pt]{article}
\usepackage[margin=1.0in]{geometry}
\usepackage{pbox}
\begin{document}

I work with Prof. Tim Bretl’s research group, which operates in the intersection
of robotics and neuroscience. One of their major projects is affordable and
accessible prosthetic devices. To that end, they’ve designed a 3D printed
prosthetic hand geared towards use by people in third world countries. The hand
features actuators that allow for different grips, and is controlled by nerve
signals near the prosthesis. The goal of this project is to augment the current
prosthetic hand with a stereo camera that can be used to detect objects the user
intends to grasp and automatically have the hand assume the appropriate grip.
This project is motivated by a current deficiency in prosthetic hand designs
that require the user to command the hand to assume a specific grip for grasping
an intended object. The commanding process makes the act of grasping an object
unnatural and time consuming, often motivating users to use their non-prosthetic
arm rather than the prosthesis. We plan to improve this deficiency by having
computer vision algorithms detect the intended object and choose an appropriate
grip for the prosthetic hand automatically. \newline

The camera will be added to the current prosthetic hand design. A camera
compartment will be added near the palm of the hand and the thumb servo
compartment has been shifted outward slightly. The camera’s viewing direction
will be chosen to face in the direction of intended grasping objects; i.e. when
grasping an object, the palm generally faces the object. See the figure below.
\newline

The chosen camera is a DUO mini lx. This is a stereo camera that can provide RGB
and depth images in indoor and outdoor scenes. The motivation for a depth camera
is two-fold: i) we don’t want the hand to constantly adjust to different grips
based on what the camera sees, so the depth allows us to know if we are close to
an object, and ii) the 3D shape of the object can potentially be leveraged for
grip selection. \newline

Using the RGB and depth information from the camera, we want to
locate the object the user intends to grasp. There are many ways that this could
be done. Examples include (i) identifying the contour, (ii) decomposing the
object into primitives, (iii) building a voxel representation from depth, (iv)
and matching against a bank of objects with machine learning. \newline

Once the intended object has been segmented in the image/video, we want to
choose the appropriate grip for grasping. The grip depends on the object’s
geometry, and the mapping between geometry and grip must be determined
experimentally. \newline

The experimental component of this project will be determining the association
of object geometries to grip type. We will put the DUO mini in a glove and take
video footage of people grasping various objects. This footage will be annotated
with the grasps that are appropriate for given objects. Then we have to test the
efficacy of object detection and segmentation algorithm by comparing chosen
prosthesis grips with the grips chosen by a person in the original annotated
footage. \newline

We expect that automating grip selection will yield better and faster results
than the alternative. We expect the stereo camera to assist the process in two
ways: 1) using the provided depth info to know when we are near an object, and
2) using both RGB and Depth data to recognize the intended object. We expect
these benefits to be clear from results showing that a user can more quickly
grasp objects than before. We also plan to provide a video demonstration of the
process and publish the result in an appropriate conference or journal. \newline

The expected timeline is as follows: \newline
\begin{tabular}{| l | l |}
\hline
Aug 24 & Buy camera \\ \hline
Sep 14 & \pbox{20cm}{a) Camera driver to collect and save RGBD images \\
                     b) Augment current hand with camera compartment  } \\ \hline
Sep 28 & Collect grasping data \\ \hline
Feb 5 & \pbox{20cm}{a) Collect grasping data \\
                    b) Annotate grasping data \\
                    c) Vision algorithm for segmenting objects} \\ \hline
Feb 17 & \pbox{20cm}{a) Vision algorithm for segmenting objects \\
                     b) Picking grip} \\ \hline
Feb 24 & \pbox{20cm}{a) Picking a grip \\
                     b) Experiments} \\ \hline
Mar 2 & \pbox{20cm}{a) Experiments \\
                      b) Write paper} \\ \hline
Mar 9 & \pbox{20cm}{a) Write paper \\
                    b) Video demo} \\ \hline
Mar 15 & Submit paper \\ \hline
\end{tabular}

\end{document}

